% LyX 2.2.2 created this file.  For more info, see http://www.lyx.org/.
%% Do not edit unless you really know what you are doing.
\documentclass[ruled]{article}
\usepackage{courier}
\usepackage[T1]{fontenc}
\usepackage[latin9]{inputenc}
\usepackage[letterpaper]{geometry}
\geometry{verbose}
\usepackage{color}
\usepackage{url}
\usepackage{algorithm2e}
\usepackage{amsmath}
\usepackage{amssymb}
\usepackage[unicode=true,
 bookmarks=false,
 breaklinks=false,pdfborder={0 0 1},backref=section,colorlinks=true]
 {hyperref}

\makeatletter

%%%%%%%%%%%%%%%%%%%%%%%%%%%%%% LyX specific LaTeX commands.
\providecommand{\LyX}{\texorpdfstring%
  {L\kern-.1667em\lower.25em\hbox{Y}\kern-.125emX\@}
  {LyX}}
%% Special footnote code from the package 'stblftnt.sty'
%% Author: Robin Fairbairns -- Last revised Dec 13 1996
\let\SF@@footnote\footnote
\def\footnote{\ifx\protect\@typeset@protect
    \expandafter\SF@@footnote
  \else
    \expandafter\SF@gobble@opt
  \fi
}
\expandafter\def\csname SF@gobble@opt \endcsname{\@ifnextchar[%]
  \SF@gobble@twobracket
  \@gobble
}
\edef\SF@gobble@opt{\noexpand\protect
  \expandafter\noexpand\csname SF@gobble@opt \endcsname}
\def\SF@gobble@twobracket[#1]#2{}

%%%%%%%%%%%%%%%%%%%%%%%%%%%%%% Textclass specific LaTeX commands.
\newenvironment{lyxcode}
{\par\begin{list}{}{
\setlength{\rightmargin}{\leftmargin}
\setlength{\listparindent}{0pt}% needed for AMS classes
\raggedright
\setlength{\itemsep}{0pt}
\setlength{\parsep}{0pt}
\normalfont\ttfamily}%
 \item[]}
{\end{list}}

\@ifundefined{date}{}{\date{}}
%%%%%%%%%%%%%%%%%%%%%%%%%%%%%% User specified LaTeX commands.
\definecolor{mygreen}{rgb}{0,0.6,0}
\definecolor{mygray}{rgb}{0.5,0.5,0.5}
\definecolor{mymauve}{rgb}{0.58,0,0.82}

\makeatother

\usepackage{listings}
\lstset{backgroundcolor={\color{white}},
basicstyle={\footnotesize\ttfamily},
breakatwhitespace=false,
breaklines=true,
captionpos=b,
commentstyle={\color{mygreen}},
deletekeywords={...},
escapeinside={\%*}{*)},
extendedchars=true,
frame=shadowbox,
keepspaces=true,
keywordstyle={\color{blue}},
language=Python,
morekeywords={*,...},
numbers=none,
numbersep=5pt,
numberstyle={\tiny\color{mygray}},
rulecolor={\color{black}},
showspaces=false,
showstringspaces=false,
showtabs=false,
stepnumber=1,
stringstyle={\color{mymauve}},
tabsize=2}
\begin{document}
\global\long\def\reals{\mathbf{R}}
 \global\long\def\integers{\mathbf{Z}}
\global\long\def\naturals{\mathbf{N}}
 \global\long\def\rationals{\mathbf{Q}}
\global\long\def\ca{\mathcal{A}}
\global\long\def\cb{\mathcal{B}}
 \global\long\def\cc{\mathcal{C}}
 \global\long\def\cd{\mathcal{D}}
\global\long\def\ce{\mathcal{E}}
\global\long\def\cf{\mathcal{F}}
\global\long\def\cg{\mathcal{G}}
\global\long\def\ch{\mathcal{H}}
\global\long\def\ci{\mathcal{I}}
\global\long\def\cj{\mathcal{J}}
\global\long\def\ck{\mathcal{K}}
\global\long\def\cl{\mathcal{L}}
\global\long\def\cm{\mathcal{M}}
\global\long\def\cn{\mathcal{N}}
\global\long\def\co{\mathcal{O}}
\global\long\def\cp{\mathcal{P}}
\global\long\def\cq{\mathcal{Q}}
\global\long\def\calr{\mathcal{R}}
\global\long\def\cs{\mathcal{S}}
\global\long\def\ct{\mathcal{T}}
\global\long\def\cu{\mathcal{U}}
\global\long\def\cv{\mathcal{V}}
\global\long\def\cw{\mathcal{W}}
\global\long\def\cx{\mathcal{X}}
\global\long\def\cy{\mathcal{Y}}
\global\long\def\cz{\mathcal{Z}}
\global\long\def\ind#1{1(#1)}
\global\long\def\pr{\mathbb{P}}

\global\long\def\ex{\mathbb{E}}
\global\long\def\var{\textrm{Var}}
\global\long\def\cov{\textrm{Cov}}
\global\long\def\sgn{\textrm{sgn}}
\global\long\def\sign{\textrm{sign}}
\global\long\def\kl{\textrm{KL}}
\global\long\def\law{\mathcal{L}}
\global\long\def\eps{\varepsilon}
\global\long\def\convd{\stackrel{d}{\to}}
\global\long\def\eqd{\stackrel{d}{=}}
\global\long\def\del{\nabla}
\global\long\def\loss{\ell}
\global\long\def\tr{\operatorname{tr}}
\global\long\def\trace{\operatorname{trace}}
\global\long\def\diag{\text{diag}}
\global\long\def\rank{\text{rank}}
\global\long\def\linspan{\text{span}}
\global\long\def\proj{\text{Proj}}
\global\long\def\argmax{\operatornamewithlimits{arg\, max}}
\global\long\def\argmin{\operatornamewithlimits{arg\, min}}
\global\long\def\bfx{\mathbf{x}}
\global\long\def\bfy{\mathbf{y}}
\global\long\def\bfl{\mathbf{\lambda}}
\global\long\def\bfm{\mathbf{\mu}}
\global\long\def\calL{\mathcal{L}}
\global\long\def\vw{\boldsymbol{w}}
\global\long\def\vx{\boldsymbol{x}}
\global\long\def\vxi{\boldsymbol{\xi}}
\global\long\def\valpha{\boldsymbol{\alpha}}
\global\long\def\vbeta{\boldsymbol{\beta}}
\global\long\def\vsigma{\boldsymbol{\sigma}}
\global\long\def\vmu{\boldsymbol{\mu}}
\global\long\def\vtheta{\boldsymbol{\theta}}
\global\long\def\vd{\boldsymbol{d}}
\global\long\def\vs{\boldsymbol{s}}
\global\long\def\vt{\boldsymbol{t}}
\global\long\def\vh{\boldsymbol{h}}
\global\long\def\ve{\boldsymbol{e}}
\global\long\def\vf{\boldsymbol{f}}
\global\long\def\vg{\boldsymbol{g}}
\global\long\def\vz{\boldsymbol{z}}
\global\long\def\vk{\boldsymbol{k}}
\global\long\def\va{\boldsymbol{a}}
\global\long\def\vb{\boldsymbol{b}}
\global\long\def\vv{\boldsymbol{v}}
\global\long\def\vy{\boldsymbol{y}}


\title{Machine Learning and Computational Statistics\\
Homework 6: Ensemble Methods {[}DRAFT - More Problems Coming{]}}
\author{Zhuoru Lin}
\maketitle
\textbf{Due: TBD in week after test, at 10pm (Submit via Gradescope)}

\textbf{Instructions}: Your answers to the questions below, including
plots and mathematical work, should be submitted as a single PDF file.
It's preferred that you write your answers using software that typesets
mathematics (e.g. \LaTeX{}, \LyX{}, or MathJax via iPython), though
if you need to you may scan handwritten work. You may find the \href{https://github.com/gpoore/minted}{minted}
package convenient for including source code in your \LaTeX{} document.
If you are using \LyX{}, then the \href{https://en.wikibooks.org/wiki/LaTeX/Source_Code_Listings}{listings}
package tends to work better.

%%%%%%%%%%%%%%%%%%%%%%%%%%%%%%%%%%%%%%%%%%%%%%%%%%%%%%%%%%%%%%%%
%1 Gradient Boosting Machine
%%%%%%%%%%%%%%%%%%%%%%%%%%%%%%%%%%%%%%%%%%%%%%%%%%%%%%%%%%%%%%%%
\section{Gradient Boosting Machines}

Recall the general gradient boosting algorithm\footnote{Besides the lecture slides, you can find an accessible discussion
of this approach in \url{http://www.saedsayad.com/docs/gbm2.pdf},
in one of the original references \url{http://statweb.stanford.edu/~jhf/ftp/trebst.pdf},
and in this review paper \url{http://web.stanford.edu/~hastie/Papers/buehlmann.pdf}. }, for a given loss function $\ell$ and a hypothesis space $\cf$
of regression functions (i.e. functions mapping from the input space
to $\reals$): 
\begin{enumerate}
\item Initialize $f_{0}(x)=0$. 
\item For $m=1$ to $M$:

\begin{enumerate}
\item Compute: 
\[
{\bf g}_{m}=\left(\left.\frac{\partial}{\partial f(x_{j})}\sum_{i=1}^{n}\ell\left(y_{i},f(x_{i})\right)\right|_{f(x_{i})=f_{m-1}(x_{i}),\,i=1,\ldots,n}\right)_{j=1}^{n}
\]
\item Fit regression model to $-{\bf g}_{m}$: 
\[
h_{m}=\argmin_{h\in\cf}\sum_{i=1}^{n}\left(\left(-{\bf g}_{m}\right)_{i}-h(x_{i})\right)^{2}.
\]
\item Choose fixed step size $\nu_{m}=\nu\in(0,1]$, or take 
\[
\nu_{m}=\argmin_{\nu>0}\sum_{i=1}^{n}\ell\left(y_{i},f_{m-1}(x_{i})+\nu h_{m}(x_{i})\right).
\]
\item Take the step: 
\[
f_{m}(x)=f_{m-1}(x)+\nu_{m}h_{m}(x)
\]
\end{enumerate}
\item Return $f_{M}$. 
\end{enumerate}
In this problem we'll derive two special cases of the general gradient
boosting framework: $L_{2}$-Boosting and BinomialBoost. 


\begin{enumerate}
%%%%%%%%%%%%%%%%%%%%%%%%%%%%%%%%%%%%%%%%%%%%%%%%%%%%%%%%%%%%%%%%
%1.1 
%%%%%%%%%%%%%%%%%%%%%%%%%%%%%%%%%%%%%%%%%%%%%%%%%%%%%%%%%%%%%%%%
\item Consider the regression framework, where $\cy=\reals$. Suppose our
loss function is given by 
\[
\ell(\hat{y},y)=\frac{1}{2}\left(\hat{y}-y\right)^{2},
\]
and at the beginning of the $m$'th round of gradient boosting, we
have the function $f_{m-1}(x)$. Show that the $h_{m}$ chosen as
the next basis function is given by 
\[
h_{m}=\argmin_{h\in\cf}\sum_{i=1}^{n}\left[\left(y_{i}-f_{m-1}(x_{i})\right)-h(x_{i})\right]^{2}.
\]
In other words, at each stage we find the weak prediction function
$h_{m}\in\cf$ that is the best fit to the residuals from the previous
stage. {[}Hint: Once you understand what's going on, this is a pretty
easy problem.{]} 
%%%%%%%%%%%%%%%%%%%%%%%%%%%%%%%%%%%%%%%%%%%%%%%%%%%%%%%%%%%%%%%%
% 1.1 Answer
%%%%%%%%%%%%%%%%%%%%%%%%%%%%%%%%%%%%%%%%%%%%%%%%%%%%%%%%%%%%%%%%
\\
\noindent\rule{14.3cm}{2pt}
\textbf{Answer:}\\
The negative gradient direction of $\ell$ can be calculated by:
\[
-\frac{\partial \ell(\hat{y},y)}{\partial \hat{y}} = -(\hat{y}-y) = y-\hat{y}=y-f(x)
\]
$h_m$ is chosen to minimize the square error with $-\frac{\partial \ell(\hat{y},y)}{\partial \hat{y}}$, therefore:
\[
h_{m}=\argmin_{h\in\cf}\sum_{i=1}^{n}\left[\left(y_{i}-f_{m-1}(x_{i})\right)-h(x_{i})\right]^{2}.
\]
%%%%%%%%%%%%%%%%%%%%%%%%%%%%%%%%%%%%%%%%%%%%%%%%%%%%%%%%%%%%%%%%
% 1.2
%%%%%%%%%%%%%%%%%%%%%%%%%%%%%%%%%%%%%%%%%%%%%%%%%%%%%%%%%%%%%%%%
\pagebreak
\item Now let's consider the classification framework, where $\cy=\left\{ -1,1\right\} $.
In lecture, we noted that AdaBoost corresponds to forward stagewise
additive modeling with the exponential loss, and that the exponential
loss is not very robust to outliers (i.e. outliers can have a large
effect on the final prediction function). Instead, let's consider
the logistic loss 
\[
\ell(m)=\ln\left(1+e^{-m}\right),
\]
where $m=yf(x)$ is the margin. Similar to what we did in the $L_{2}$-Boosting
question, write an expression for $h_{m}$ as an argmin over $\cf$.
%%%%%%%%%%%%%%%%%%%%%%%%%%%%%%%%%%%%%%%%%%%%%%%%%%%%%%%%%%%%%%%%
% 1.2 Answer
%%%%%%%%%%%%%%%%%%%%%%%%%%%%%%%%%%%%%%%%%%%%%%%%%%%%%%%%%%%%%%%%
\\
\noindent\rule{14.3cm}{2pt}\\
\textbf{Answer:}
The negative gradient direction is calculated by:
\[
-\frac{\partial \ell(y,f(x))}{\partial f(x)}=\frac{ye^{-yf(x)}}{1+e^{-yf(x)}}
\]
$h_m$ is chosen by:
\[
\argmin_{h \in \cf} \sum_{i=1}^{n}\left[ \frac{ye^{-yf(x)}}{1+e^{-yf(x)}}-h(x_i)  \right]^2
\]
\end{enumerate}

\pagebreak
\section{From Margins to Conditional Probabilities\protect\footnote{This problem is based on Section 7.5.3 of Schapire and Freund's book
\emph{Boosting: Foundations and Algorithms}.}}

Let's consider the classification setting, in which $\left(x_{1},y_{1}\right),\ldots,\left(x_{n},y_{n}\right)\in\cx\times\left\{ -1,1\right\} $
are sampled i.i.d. from some unknown distribution. For a prediction
function $f:\cx\to\reals$, we define the \textbf{margin }on an example
$\left(x,y\right)$ to be $m=yf(x)$. Since our class predictions
are given by $\sign(f(x))$, we see that a prediction is correct iff
$m(x)>0$. We have said we can interpret the magnitude of the margin
$\left|m(x)\right|$ as a measure of confidence. However, it is not
clear what the ``units'' of the margin are, so it is hard to interpret
the magnitudes beyond saying one prediction is more or less confident
than another. In this problem, we investigate how we can translate
the margin into a conditional probability, which is much easier to
interpret. In other words, we are looking for a mapping $m(x)\mapsto p(y=1\mid x)$. 

In this problem we will consider margin-based losses. A loss function
is a \textbf{margin-based loss }if it can be written in terms of the
margin $m=yf(x)$. We are interested in how we can go from an empirical
risk minimizer of a margin loss, $\hat{f}=\argmin_{f\in\cf}\sum_{i=1}^{n}\ell\left(y_{i}f(x_{i})\right)$,
to a conditional probability estimator $\hat{\pi}(x)\approx p(y=1\mid x)$.
Our approach will be to try to find a way to use the Bayes prediction
function\footnote{In this context, the Bayes prediction function is often referred to
as the ``population minimizer.'' In our case, ``population'' referes
to the fact that we are minimizing with respect to the true distribution,
rather than a sample. The term ``population'' arises from the context
where we are using a sample to approximate some statistic of an entire
population (e.g. a population of people or trees).} $f^{*}=\argmin_{f}\ex_{x,y}\left[\ell(yf(x)\right]$ to get the true
conditional probability $p(y=1\mid x$), and then apply the same mapping
to the empirical risk minimizer. While there is plenty that can go
wrong with this ``plug-in'' approach (primarily, the empirical risk
minimizer from a hypothesis space $\cf$ may be a poor estimate for
the Bayes prediction function), it is at least well-motivated, and
it can work well in practice. And \textbf{please note} that we can
do better than just hoping for success: if you have enough validation
data, you can directly assess how well ``calibrated'' the predicted
probabilities are. This blog post has some discussion of calibration
plots: \href{https://jmetzen.github.io/2015-04-14/calibration.html}{https://jmetzen.github.io/2015-04-14/calibration.html}. 

It turns out it is straightforward to find the Bayes prediction function
$f^{*}$ for margin losses, at least in terms of the data-generating
distribution: For any given $x\in\cx$, we'll find the best possible
prediction $\hat{y}$. This will be the $\hat{y}$ that minimizes
\[
\ex_{y}\left[\ell\left(y\hat{y}\right)\mid x\right].
\]
If we can calculate this $\hat{y}$ for all $x\in\cx$, then we will
have determined $f^{*}(x)$. We will simply take
\[
f^{*}(x)=\argmin_{\hat{y}}\ex_{y}\left[\ell\left(y\hat{y}\right)\mid x\right].
\]

Below we'll calculate $f^{*}$ for several loss functions. It will
be convenient to let $\pi(x)=\pr\left(y=1\mid x\right)$ in the work
below.
\begin{enumerate}
%%%%%%%%%%%%%%%%%%%%%%%%%%%%%%%%%%%%%%%%%%%%%%%%%%%%%%%%%%%%%%%%
% 2.1 
%%%%%%%%%%%%%%%%%%%%%%%%%%%%%%%%%%%%%%%%%%%%%%%%%%%%%%%%%%%%%%%%
\item Write $\ex_{y}\left[\ell\left(yf(x)\right)\mid x\right]$ in terms
of $\pi(x)$ and $\ell\left(f(x)\right)$. {[}Hint: Use the fact that
$y\in\left\{ -1,1\right\} $.{]}
%%%%%%%%%%%%%%%%%%%%%%%%%%%%%%%%%%%%%%%%%%%%%%%%%%%%%%%%%%%%%%%%
% 2.1 Answer
%%%%%%%%%%%%%%%%%%%%%%%%%%%%%%%%%%%%%%%%%%%%%%%%%%%%%%%%%%%%%%%%
\\
\noindent\rule{14.5cm}{2pt}\\
\textbf{Answer:}\\
\[
\pi(x)\ell(f(x))+(1-\pi(x)\ell(-f(x))
\]
\pagebreak
%%%%%%%%%%%%%%%%%%%%%%%%%%%%%%%%%%%%%%%%%%%%%%%%%%%%%%%%%%%%%%%%
% 2.2
%%%%%%%%%%%%%%%%%%%%%%%%%%%%%%%%%%%%%%%%%%%%%%%%%%%%%%%%%%%%%%%%
\item Show that the Bayes prediction function $f^{*}(x)$ for the exponential
loss function $\ell\left(y,f(x)\right)=e^{-yf(x)}$ is given by 
\[
f^{*}(x)=\frac{1}{2}\ln\left(\frac{\pi(x)}{1-\pi(x)}\right)
\]
and, given the Bayes prediction function $f^{*}$, we can recover
the conditional probabilities by
\[
\pi(x)=\frac{1}{1+e^{-2f^{*}(x)}}.
\]
{[}Hint: Differentiate the expression in the previous problem with
respect to $f(x)$. To make things a little less confusing, and also
to write less, you may find it useful to change variables a bit: Fix
an $x\in\cx$. Then write $p=\pi(x)$ and $\hat{y}=f(x)$. After substituting
these into the expression you had for the previous problem, you'll
want to find $\hat{y}$ that minimizes the expression. Use differential
calculus. Once you've done it for a single $x$, it's easy to write
the solution as a function of $x$.{]} 
%%%%%%%%%%%%%%%%%%%%%%%%%%%%%%%%%%%%%%%%%%%%%%%%%%%%%%%%%%%%%%%%
% 2.2 Answer
%%%%%%%%%%%%%%%%%%%%%%%%%%%%%%%%%%%%%%%%%%%%%%%%%%%%%%%%%%%%%%%%
\\
\noindent\rule{14.3cm}{2pt}\\
\textbf{Answer:}\\
Let $R$ be the Bayes risk:
\begin{align}
	R = \ex_{y}\left[\ell\left(yf(x)\right)\mid x\right]&=\pi(x)e^{-f(x)}+(1-\pi(x))e^{f(x)} \nonumber\\
	\frac{\partial R}{f(x)}&=-\pi(x)e^{-f(x)}+(1+\pi(x))e^{f(x)}
\end{align}
The Bayes risk minimizer should have $\frac{\partial R}{f(x)}=0$.
\begin{align}
	\frac{\partial R}{f(x)}=0 \iff (1+\pi(x))e^{f^{*}(x)} &= \pi(x)e^{-f^{*}(x)} \nonumber\\
	\frac{e^{f^{*}(x)}}{e^{-f^{*}(x)}} &= \frac{\pi(x)}{1+\pi(x)} \nonumber	\\
	f^{*}(x) &= \frac{1}{2}(\frac{\pi(x)}{1-\pi(x)}) 
\end{align}

Rearrange (1) we get:
\begin{align*}
\pi(x)=\frac{1}{1+e^{-2f^{*}(x)}}.
\end{align*}

%%%%%%%%%%%%%%%%%%%%%%%%%%%%%%%%%%%%%%%%%%%%%%%%%%%%%%%%%%%%%%%%
% 2.3 
%%%%%%%%%%%%%%%%%%%%%%%%%%%%%%%%%%%%%%%%%%%%%%%%%%%%%%%%%%%%%%%%
\pagebreak
\item Show that the Bayes prediction function $f^{*}(x)$ for the logistic
loss function $\ell\left(y,f(x)\right)=\ln\left(1+e^{-yf(x)}\right)$
is given by
\[
f^{*}(x)=\ln\left(\frac{\pi(x)}{1-\pi(x)}\right)
\]
and the conditional probabilities are given by
\[
\pi(x)=\frac{1}{1+e^{-f^{*}(x)}}.
\]
 Again, we may assume that $\pi(x)\in(0,1)$.
%%%%%%%%%%%%%%%%%%%%%%%%%%%%%%%%%%%%%%%%%%%%%%%%%%%%%%%%%%%%%%%%
% 2.3 Answer
%%%%%%%%%%%%%%%%%%%%%%%%%%%%%%%%%%%%%%%%%%%%%%%%%%%%%%%%%%%%%%%%
\\
\noindent\rule{14.3cm}{2pt}\\
\textbf{Answer:}\\
Let $R$ be the Bayes risk:
\begin{align}
R = \ex_{y}\left[\ell\left(yf(x)\right)\mid x\right]&=\pi(x)(1+e^{-f(x)})+(1-\pi(x))(1+e^{f(x)}) \nonumber\\
\frac{\partial R}{f(x)}&=-\pi(x)e^{-f(x)}+(1+\pi(x))e^{f(x)}
\end{align}

Notice that (3) and (2) are exactly the same thus we have
\[
f^{*}(x)=\ln\left(\frac{\pi(x)}{1-\pi(x)}\right)
\]
Rearrange we get:
\begin{align*}
\pi(x)=\frac{1}{1+e^{-2f^{*}(x)}}.
\end{align*}

So Logistic lost and exponential lost have the exactly same population minimizer.
\pagebreak
\item {[}Optional{]} Show that the Bayes prediction function $f^{*}(x)$
for the hinge loss function $\ell\left(y,f(x)\right)=\max\left(0,1-yf(x)\right)$
is given by
\[
f^{*}(x)=\sign\left(\pi(x)-\frac{1}{2}\right).
\]
Note that it is impossible to recover $\pi(x)$ from $f^{*}(x)$ in
this scenario. However, in practice we work with an empirical risk
minimizer, from which we may still be able to recover a reasonable
estimate for $\pi(x)$. An early approach to this problem is known
as ``Platt scaling'': \href{https://en.wikipedia.org/wiki/Platt_scaling}{https://en.wikipedia.org/wiki/Platt\_{}scaling}.
\end{enumerate}

\section{AdaBoost Actually Works {[}Optional{]}}

\global\long\def\mathbbm{}


\subsection*{Introduction}

Given training set $D=\{(x_{1},y_{1}),\dots,(x_{n},y_{n})\},$ where
$y_{i}$'s are either $+1$ or $-1$, suppose we have a weak learner
$G_{t}$ at time $t$ and we will perform $T$ rounds of AdaBoost.
Initialize observation weights uniformly by setting $W^{1}=(w_{1}^{1},\dots,w_{n}^{1})$
with $w_{i}^{1}=1/n$ for $i=1,2,\dots,n.$ For $t=1,2,\dots,n$: 
\begin{enumerate}
\item Fit the weak learner $G_{t}$ at time $t$ to training set $D$ with
weighting $W^{t}$.
\item Compute the weighted misclassification error: $\text{err}_{t}=\sum_{D}w_{i}^{t}\ind{G_{t}(x_{i})\neq y_{i}}/\sum_{i}w_{i}^{t}$ 
\item Compute the contribution coefficient for the weak learner: $\alpha_{t}=\frac{1}{2}\log(\frac{1}{\text{err}_{t}}-1)$ 
\item Update the weights: $w_{i}^{t+1}=w_{i}^{t}\exp(-\alpha_{t}y_{i}G_{t}(x_{i}))$ 
\end{enumerate}
After $T$ steps, the cumulative contributions of weak learners is
$G(x)=\text{sign}(\sum_{t=1}^{T}\alpha_{t}G_{t}(x))$ as the final
output. We will prove that with a reasonable weak learner the error
of the output decreases exponentially fast with the number of iterations.

\subsection*{Exponential bound on the training loss}

More precisely, we will show that the training error $L(G,D)=\frac{1}{n}\sum_{i=1}^{n}\mathbbm{1}_{\{G(x_{i})\neq y_{i}\}}\leq\exp(-2\gamma^{2}T)$
where the error of the weak learner is less than $1/2-\gamma$ for
some $\gamma>0$. To start, let's denote two cumulative variables:
the output at time $t$ as $f_{t}=\sum_{s\leq t}\alpha_{s}G_{s}$
and $Z_{t}=\frac{1}{n}\sum_{i=1}^{n}\exp(-y_{i}f_{t}(x_{i}))$.
\begin{enumerate}
\item For any function $g$, show that $\mathbbm{1}_{\{g(x)\neq y\}}<\exp{(-yg(x))}$.\\
\item Use this to show $L(G,D)<Z_{T}$\\
\item Show that $w_{i}^{t+1}=\exp(-y_{i}f_{t}(x_{i}))$\\
\item Use part 3 to show $\frac{Z_{t+1}}{Z_{t}}=2\sqrt{\text{err}_{t+1}(1-\text{err}_{t+1})}$
(Hint: use the definition of weight updates and separate the sum on
where $G_{t}$ is equal to 1 and $-1$.)\\
\item Show that the function $g(a)=a(1-a)$ is monotonically increasing
on $[0,1/2]$. Show that $1-a\leq\exp(-a)$. And use the assumption
on the weak learner to show that $\frac{Z_{t+1}}{Z_{t}}\leq\exp(-2\gamma^{2})$\\
\item Conclude the proof!\\
\end{enumerate}

\section{AdaBoost is FSAM With Exponential Loss {[}Optional{]}}

The AdaBoost score function $G(x)=\sum_{t=1}^{T}\beta_{t}G_{t}(x)$
is a linear combination (actually a conic combination) of functions.
(The prediction function is, of course, the sign of the score function.)
Forward stagewise additive modeling (FSAM) is another approach to
fitting a function of this form. 

In FSAM, we have a base hypothesis space $\ch$ of real-valued functions
$h:\cx\to\reals$ and a loss function $\ell\left(y,\hat{y}\right)$.
In FSAM, we attempt to find a linear combination of $h$'s in $\ch$
that minimize the empirical risk. The procedure initializes $f_{0}(x)=0$,
and then repeats the following steps for $t=1,\dots,T$:
\begin{enumerate}
\item $(\beta_{t},h_{t})=\text{argmin}_{\beta\in\reals,h\in\ch}\sum_{i=1}^{n}\ell(y_{i},f_{t-1}(x_{i})+\beta h(x_{i}))$ 
\item $f_{t}(x)=f_{t-1}(x)+\beta_{t}h_{t}(x)$ 
\end{enumerate}

\subsection*{Exponential loss and AdaBoost}

Consider a generic input space $\cx$, the classification outcome
space $\cy=\left\{ -1,1\right\} $, the exponential loss function
$\ell(y,f(x))=\exp(-yf(x))$, and an arbitrary base hypothesis space
$\ch$ consisting of $\left\{ -1,1\right\} $-valued functions. We
will show that FSAM in this setting is equivalent to a version of
AdaBoost (Algorithm \ref{alg:Exact-Adaboost-Algorithm}) described
below. To get this equivalence, we either need to assume that FSAM
chooses nonnegative step sizes, i.e. $\beta_{t}\ge0$, or we need
to assume that $\ch$ is symmetric, in the sense that if $h\in\ch$,
then $-h\in\ch$ as well. 
\begin{enumerate}
\item Write the first step of FSAM using the exponential loss function.
In particular, show that the FSAM optimization problem can be written
as a minimization of a weighted exponential loss of the step $\beta h$:
\[
(\beta_{t},h_{t})=\text{argmin}_{\beta,h\in\ch}\left(\frac{1}{\sum_{i=1}^{n}w_{i}^{t}}\right)\sum_{i=1}^{n}w_{i}^{t}\exp(-y_{i}\beta h(x_{i})),
\]
where $w_{i}^{t}=\exp(-y_{i}f_{t-1}(x_{i}))$. (Note that for any
$t$, if we rescale each of $w_{1}^{t},\ldots,w_{n}^{t}$ by the same
constant factor, there is no effect on the $\argmin$. Thus the first
factor $\left(\sum_{i=1}^{n}w_{i}^{t}\right)^{-1}$ can be dropped.
However, we keep it so we can refer to the expression as a \textbf{weighted
mean}.) 
\item \global\long\def\err{\text{err}}
 Define the weighted $0/1$ error of $h$ at round $t$ to be
\begin{eqnarray*}
\err_{t}(h) & = & \left(\frac{1}{\sum_{i=1}^{n}w_{i}^{t}}\right)\sum_{i=1}^{n}w_{i}^{t}\ind{y_{i}\neq h(x_{i})}.
\end{eqnarray*}
 (It's the weights that are specific to round $t$.) Show that the
weighted exponential loss at round $t$ can be written in terms of
the weighted $0/1$ error. Specifically, show that
\[
\left(\frac{1}{\sum_{i=1}^{n}w_{i}^{t}}\right)\sum_{i=1}^{n}w_{i}^{t}\exp(-\beta y_{i}h(x_{i}))=e^{-\beta}+\left(e^{\beta}-e^{-\beta}\right)\err_{t}(h).
\]
{[}Hint: Use indicators $\ind{h(x_{i})\neq y_{i}}$ and $\ind{h(x_{i})=y_{i}}$
to split the summand on the LHS into pieces. Each piece simplifies,
since $y_{i},h(x_{i})\in\left\{ -1,1\right\} $. Then note that $\ind{h(x_{i})=y_{i}}=1-\ind{h(x_{i})\neq y_{i}}$.{]}
\item We now would like to show that for any fixed ``step size'' $\beta$,
the optimal ``step direction'' $h$, for which $\beta h$ minimizes
the weighted exponential loss, can be found by minimizing the weighted
$0/1$ error of $h$. But more precisely, show that if $\beta\ge0$
then 
\[
\text{argmin}_{h\in\ch}\left(\frac{1}{\sum_{i=1}^{n}w_{i}^{t}}\right)\sum_{i=1}^{n}w_{i}^{t}\exp(-\beta y_{i}h(x_{i}))=\text{argmin}_{h\in\ch}\err_{t}(h).
\]
Also show that if $\beta<0$ then
\[
\text{argmin}_{h\in\ch}\left(\frac{1}{\sum_{i=1}^{n}w_{i}^{t}}\right)\sum_{i=1}^{n}w_{i}^{t}\exp(-\beta y_{i}h(x_{i}))=\text{argmin}_{h\in\ch}\err_{t}(-h).
\]
\\
 
\item Show that if $\ch$ is symmetric, in the sense that $h\in\ch$ implies
$-h\in\ch$, then there is always an optimal FSAM step $\left(\beta_{t},h_{t}\right)$
with $\beta_{t}\ge0$. Thus if we assume that either $\ch$ is symmetric
or FSAM chooses nonnegative step sizes, then we can conclude that
\[
h_{t}=\text{argmin}_{h\in\ch}\err_{t}(h)
\]
is a solution to $h_{t}$ in the minimization problem in the first
part, and thus is the FSAM step direction in round $t$.
\item Now that we've found $h_{t}$, show that the corresponding optimal
step size is given by $\beta_{t}=\frac{1}{2}\log\left(\frac{1-\err_{t}}{\err_{t}}\right)$,
where we let $\err_{t}=\err_{t}(h_{t})$ as a shorthand. {[}Hint:
You'll need to use some differential calculus. Show that what you've
found is a minimum by showing that the function you're differentiating
is convex.{]}\\
\item Show that 
\begin{eqnarray*}
w_{i}^{t+1} & = & \begin{cases}
e^{-\beta_{t}}w_{i}^{t} & \text{if }y_{i}=h_{t}(x_{i})\\
e^{-\beta_{t}}w_{i}^{t}e^{2\beta{}_{t}} & \text{otherwise,}
\end{cases}
\end{eqnarray*}
This is the weight update equation from AdaBoost. {[}Hint: First show
that $w_{i}^{t+1}=w_{i}^{t}\exp(-\beta_{t}y_{i}h_{t}(x_{i}))$. Then
write $y_{i}h_{t}(x_{i})$ in terms of the indicator function $y_{i}\neq h_{t}(x_{i})$.{]}
\item Let's introduce a specific instance of AdaBoost we'll call ``Exact
AdaBoost'', given in Algorithm \ref{alg:Exact-Adaboost-Algorithm}.
\begin{algorithm}[h]
\caption{\label{alg:Exact-Adaboost-Algorithm}Exact AdaBoost}

\begin{lyxcode}
input:~Training~set~$\cd=\left(\left(x_{1},y_{1}\right),\ldots,(x_{n},y_{n})\right)\in\cx\times\left\{ -1,1\right\} $~\\
$w_{i}^{1}=1$~for~$i=1,\ldots,n$~\#Initialize~weights~\\
for~$t=1,\ldots,T$:~\\
~~$h_{t}=\argmin_{h\in\ch}\sum_{i=1}^{n}w_{i}^{t}\ind{y_{i}\neq h(x_{i})}$~\\
~~$\err_{t}=\err_{t}(h_{t})=\left(\frac{1}{\sum_{i=1}^{n}w_{i}^{t}}\right)\sum_{i=1}^{n}w_{i}^{t}\ind{y_{i}\neq h(x_{i})}$~\\
~~$\alpha_{t}=\ln\left(\frac{1-\err_{t}}{\err_{t}}\right)$

~~$w_{i}^{t+1}=\begin{cases}
w_{i}^{t} & \text{if }y_{i}=h_{t}(x_{i})\\
w_{i}^{t}e^{\alpha_{t}} & \text{otherwise,}
\end{cases}$~~for~$i=1,\ldots,n$~

return~$f=\sum_{t=1}^{T}\alpha_{t}h_{t}$~~\#Returns~the~score~function.~~

(Predictions~are~$x\mapsto\sign(f(x))$).~\\
\end{lyxcode}
\end{algorithm}
The only difference between Exact AdaBoost and AdaBoost is that in
Exact AdaBoost, we require that the base classifier return the best
possible $h\in\ch$, while in AdaBoost we only vaguely stated that
the ``base learner fits the weighted training data'', but there
was no requirement that the result be the best possible. Indeed, since
a typical base classifier is decision trees, and it's computationally
prohibitive to find the best possible tree, Exact AdaBoost is not
usually an implementable algorithm. Show that the score functions
returned by Exact Adaboost and by FSAM (in our setting) differ only
by a constant factor, and of course the hard classifications will
be exactly the same. 
\item Suppose our ultimate goal is to find the score function returned by
FSAM after $T$ rounds in the context described above. Suppose we
only have access to an implementation of Exact AdaBoost described
in Algorithm \ref{alg:Exact-Adaboost-Algorithm}, and it returns the
score function $f(x)$. What would be the score function returned
by FSAM?
\end{enumerate}

\end{document}
